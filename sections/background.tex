\paragraph{`Classic' Plasma} \label{ch2:classic_plasma}
The initial vision of Plasma describes a mechanism which enables connecting
blockchains to a base blockchain often referred to as the `rootchain' or the
`mainchain' \cite{plasma}. The mainchain acts as the final arbitrator in
the case of disputes as shown in Figure \ref{fig:many-chains}. Trees of plasmachains forming a hierarchy of blockchains similar to the court system would also be possible such that a dispute can be escalated all the way to the supreme court (the mainchain)

%Diagram - Plasma-Summary-Many-Chains.dia
\begin{figure}[H]
    \makebox[\linewidth]{
        \scalebox{0.75}{
            \includegraphics[width=\linewidth]{figures/Plasma-Summary-Many-Chains.pdf}
        }
    }
    \caption{
        Multiple blockchains connected to a `mainchain' \cite{plasma}.
    }
    \label{fig:many-chains}
\end{figure}

A plasmachain that utilizes a potentially centralized consensus algorithm (e.g. Proof of Authority) derives its security from the mainchain's consensus algorithm. This is achieved by publishing a merklized\footnote{The transaction set of each block gets inserted in a Merkle Tree, and the merkle root of that tree gets published to a smart contract.} state root of each \textit{plasma block} to the mainchain. The transaction model used is UTXO based, similar to Bitcoin\footnote{https://en.bitcoin.it/wiki/Transaction}.

\paragraph{Exits and challenges} An \textit{exit} is the mechanism by which a user expresses
their intent to withdraw a UTXO from the plasmachain back to their account on
the mainchain. After submitting the exit of a UTXO, the user needs to wait for
a \textit{challenge period} to pass, during which other users can \textit{challenge}
their exit, by providing proof that it is invalid\footnote{This is a common technique also used in payment channels, whereby invalid attestations can be challenged and the fraudulent user gets penalized if the challenge is successful.}. 

A challenge is an attestation provided by another user, proving that the UTXO
that is being exited is invalid or has been spent. Challenges can either be
non-interactive, where the exit is instantly cancelled, or interactive where
they require a response to the challenge before the \textit{challenge period} is over. 
After the challenge period has passed, if there is no outstanding challenge, 
an exit can be finalized, giving its owner the right to withdraw the amount 
specified by the UTXO back to their mainchain account. 

Exits and interactive challenges require users to attach a bond, as collateral. If a user's exit gets challenged and the challenge is valid, the collateral gets slashed and is sent to the user who reported the fraud. This incentivizes users to act honestly, and creates a crowdsourced system of watchers who are rewarded for reporting fraudulent activity.
